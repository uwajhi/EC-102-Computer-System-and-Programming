\documentclass{beamer}
\usepackage[utf8]{inputenc}

\usepackage{amsmath}
\usepackage{graphicx}
\usepackage{url}
\usepackage{fancyvrb}
\usepackage{xcolor}

\usetheme{Antibes}
\usecolortheme{whale}
\usepackage{lmodern}

\usepackage{listings}
\usepackage{color}

\definecolor{codegreen}{rgb}{0,0.6,0}
\definecolor{codegray}{rgb}{0.5,0.5,0.5}
\definecolor{codepurple}{rgb}{0.58,0,0.82}
\definecolor{backcolour}{rgb}{0.95,0.95,0.92}

\mode<presentation>

\definecolor{orange}{HTML}{BC2E07}

\usepackage{hyperref}
\hypersetup{
    colorlinks,
    linkcolor=orange,
    urlcolor=blue
}

\lstdefinestyle{mystyle}{
    language=C++,
    basicstyle=\ttfamily\footnotesize,
    backgroundcolor=\color{backcolour},
    commentstyle=\color{codegreen},
    keywordstyle=\color{magenta},
    numberstyle=\tiny\color{codegray},
    stringstyle=\color{codepurple},
    breakatwhitespace=false,
    breaklines=true,
    captionpos=b,
    keepspaces=true,
    numbers=left,
    numbersep=5pt,
    showspaces=false,
    showstringspaces=false,
    showtabs=false,
    tabsize=2
}

\title{Lab \# 4: Programming Practice and Using Header Files}
\subtitle{EC-102 -- Computer Systems and Programming}

\author{Usama Wajhi}
\institute{School of Mechanical and Manufacturing Engineering (SMME), \\ National University of Sciences and Technology (NUST)}
\date{\today}

\begin{document}
\begin{frame}
    \titlepage
\end{frame}

\begin{frame}{Outline}
        \tableofcontents
\end{frame}

\begin{frame}[label=main]
	\frametitle{Lab Task 1}
	\section{Lab Task 1}
		\begin{large}
			{Write a program which asks the user to enter the number of gallons and displays the equivalent in cubic feet.\\(1 cubic-feet = 7.48 gallons)}
		\end{large}\\
		\hyperlink{supplemental}{\beamerbutton{code}}
\end{frame}

\begin{frame}[label=main]
	\section{Header Files}
	\frametitle{Header Files}
	\begin{itemize}
		\item \hyperlink{iomanip.h}{iomanip.h}\\
		\item \hyperlink{ctype.h}{ctype.h}\\
		%\item \hyperlink{math.h}{math.h}\\
	\end{itemize}
		
\end{frame}

\begin{frame}[label=iomanip.h]
	\frametitle{iomanip.h}
	\subsection{iomanip.h}
	The iomanip is a parameterized input output manipulator.
	\begin{itemize}
	\item setprecision(int n) – It sets the precision of floating point number to n digits.
		\item setw(int n) – It sets the width of the field of the next output to n characters. If the length of the output stream is less than n then spaces are padded. The no of spaces padded is the difference between n and length of the output stream. If the length of the output stream is less than n there will be no effect on output stream.
 	\item setbase(int n) – It sets the output representation of the number to octal, decimal or hexadecimal corresponding to the argument n which is 8 in case of octal, 10 for decimal and 16 for hexadecimal.
	\item setfill ( char c) – It sets the fill character to be the value of character c.
	
	\end{itemize}
\end{frame}

\begin{frame}[label=lt2]
	\frametitle{Lab Task 2}
	\subsubsection{Lab Task 2}
	Write a code which gives you the following output.\\
	Hint: use setw and setfill.
	\begin{figure}
            \centering
            \includegraphics[scale=0.7]{Task2}
    \end{figure}
    \hyperlink{ltcode}{\beamerbutton{code}}
\end{frame}


\begin{frame}[label=ctype.h]
	\frametitle{ctype.h}
	\subsection{ctype.h}
	\begin{itemize}
		\item The ctype.h header file of the C Standard Library declares several functions that are useful for testing and mapping characters.

		\item These functions take the int equivalent of one character as parameter and the functions return non-zero (true) if the argument c satisfies the condition described, and zero(false) if not.
	\end{itemize}
\end{frame}
\begin{frame}{ctype.h functions}
\subsubsection{ctype.h functions}

\begin{itemize}

\item isalnum:
Check if character is alphanumeric
\item isalpha: \ 
Check if character is alphabetic
\item islower: \ \ 
Check if character is lowercase letter
\item isspace: \ \ 
Check if character is a white-space 
\item isupper: \ \ 
Check if character is uppercase letter
\item isxdigit: \ \ 
Check if character is hexadecimal digit 
\end{itemize}
\end{frame}

\begin{frame}[fragile]{Example}
\subsubsection{Example}
\lstset{style=mystyle}
        \begin{lstlisting}
// takes a single character (a letter) as an argument and returns a non-zero integer if the letter is lowercase, or zero if it is uppercase
#include <iostream>
#include <ctype.h>
using namespace std;

int main()
{
    char letter;
    cin >> letter;
    cout << islower(letter) << endl;
    return 0;
}
\end{lstlisting}

\end{frame}

%\begin{frame}[label=math.h]
%	\frametitle{math.h}
%	\subsection{math.h}
%	
%	
%	
%	
%\end{frame}

\begin{frame}[fragile, label=supplemental]
	\frametitle{Code Lab Task 1}
	\lstset{style=mystyle}
        \begin{lstlisting}
#include <iostream>
using namespace std;

int main()
{
	float gallons, cubic_feet;
	cout<<"enter number of gallons to convert into cubic-feet: ";
	cin >> gallons;
	cout << gallons * 0.133681 << " cubic-feet" << endl;
	system("pause");
	return 0;
}\end{lstlisting}
\hyperlink{main}{\beamerbutton{return}}
\end{frame}

\begin{frame}[label=ltcode, fragile]{Code Lab Task 2}
\lstset{style=mystyle}
        \begin{lstlisting}
#include <iostream>
#include <iomanip>
using namespace std;

int main()
{
    long p1 = 2425785, p2 = 47, p3 = 9761;
    
    cout << setfill('.') << setw(8) << "LOCATION" << setw(18) << "POPULATION" << endl 
		<< setw(8) << "Portcity" << setw(18) << p1 << endl 
		<< setw(8) << "Hightown" << setw(18) << p2 << endl 
		<< setw(8) << "Lowville" << setw(18) << p3 << endl;
    return 0;
}\end{lstlisting}
\hyperlink{lt2}{\beamerbutton{return}}

	
\end{frame}

\end{document}